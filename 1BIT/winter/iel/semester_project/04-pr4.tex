\section{Příklad 4}
% Jako parametr zadejte skupinu (A-H)
\ctvrtyZadani{H}
\subsubsection{Vzorce pro impedanci a uhlovou frekvenci}
\begin{align*}
\omega &= 2 \pi f\\
Z_C &= \frac{-j}{\omega C}\\
Z_L &= j \omega L
\end{align*}
\subsubsection{II Kir. Z. pro rovnice proudu ve smyčkách}
\begin{align*}
I_A &: U_{L1} + U_1 + U_{C2} + U_{R1} = 0\\
I_B &: U_{R1} + U_2 + U_{C1} = 0\\
I_C &: U_{C2} + U_{L2} + U_{R2} - U_2 = 0
\end{align*}
\begin{align*}
I_A &: I_A (Z_{L1} + Z_{C2} + R_1) - I_BR_1 - I_CZ_{C2}\\
I_B &: I_B (R_1 + Z_{C1}) - I_AR_1 = -U_2\\
I_C &: I_C (Z_{C2} + Z_{L2} + R_2) - I_A Z_{C2} = U_2
\end{align*}
\subsubsection{Výpočet proudu pomoci matic}
\centering
\begin{align*}
    \begin{pmatrix}
    Z_{L1} + Z_{C2} + R_1 & -R_1 & -Z_{C2}\\
    -R_1 & R_1 + Z_{C1} & 0\\
    -Z_{C2} & 0 & Z_{C2} + Z_{L2} + R_2
    \end{pmatrix}
    \times
    \begin{pmatrix}
    I_A\\
    I_B\\
    I_C
    \end{pmatrix}
    &=
    \begin{pmatrix}
    -U_1\\
    -U_2\\
    U_2\\
    \end{pmatrix}
    \\ \\
    \begin{pmatrix}
    10 + 71.5713j & -10 & 23.9391\\
    -10 & 10 - 10.8085j & 0\\
    23.9331j & 0 & 10 + 20.8346j
    \end{pmatrix}
    \times
    \begin{pmatrix}
    I_A\\
    I_B\\
    I_C
    \end{pmatrix}
    &=
    \begin{pmatrix}
    -5\\
    -6\\
    6
    \end{pmatrix}
\end{align*}
\subsubsection{Násobením inverzní matici zleva dostáváme hodnoty proudu}
\begin{align*}
I_A &= (-0.2105 + 0.2255j) A\\
I_B &= (-0.4862 - 0.3j) A\\
I_C &= (0.4099 - 0.3502j) A
\end{align*}
\subsubsection{Formuli pro napětí a fázový posun L2}
\begin{align*}
U_{L2} &= I_C \times Z_{L2}\\
\vert U_{L2}\vert &= \sqrt{ re(U_{L2})^2 + im(U_{L2})^2 }
\end{align*}
\begin{align*}
\varphi &= atan(\frac {im(U_{L2})}{re(U_{L2})})
\end{align*}
\centering
\subsubsection{Dosadíme hodnoty do vzorců}
\begin{align*}
I_{L2} &= I_C = (0.4099 - 0.3502j) A\\
U_{L2} &= I_{L2}\times Z_{L2} = (0.4099 - 0.3502j) A \times (0 + 44.7676j) A = (15.6810 + 18.3519j) V\\ \\
\boldsymbol{\varphi }&=\boldsymbol{ atan(\frac {im(U_{L2})}{re(U_{L2})}) = atan(\frac {18.3519}{15.6810})  = 0.8637^{~\circ}}\\
\boldsymbol{\vert U_{L2}\vert }&=\boldsymbol{ \sqrt{ re(U_{L2})^2 + im(U_{L2})^2 } = \sqrt{15.6810^2 + 18.3519^2} = 24.1389 V}
\end{align*}